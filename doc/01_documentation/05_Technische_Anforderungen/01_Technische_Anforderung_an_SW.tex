\chapter{Technische Anforderungen}
In diesem Kapitel werden die Technischen Voraussetzungen, für die in der Bedürfnisanalyse ermittelten Anwendungen, definiert. Dabei wird unterschieden für Allgemeine Applikationen, welche nur eine kleine Auswahl beachtet wird, und für siot.net Applikationen.

\section{Allgemeine Applikationen Anforderungen}
Die Technischen Anforderungen für die Bedürfnisse Gesundheit, Smart Home und Finanztechnologie werden in diesem Abschnitt definiert.

\subsection{Gesundheit}
Bei den Gesundheitsapplikationen ist es wichtig, dass die Smartwatch über einen Herzfrequenzmesser verfügt und die Kombination aus Gyroskop, Rotationssensor und Bewegungssensor eingebaut ist.
Diese Elemente sind erforderlich um Pulsraten eines Menschen zu messen, sowie Analysen von Bewegungen zu ermitteln.

\textbf{Benötigte Sensoren:}\\
- Gyroskop\\
- Bewegungssensor\\
- Rotationssensor\\
- Herzfrequenzmesser

\subsection{Smart Home}
Für die Steuerung von Haushaltsgeräte, Heizung oder Licht wird vorallem ein Steuerungsanzeige benötigt. Dies ermöglicht die Apparate ein- und auszuschalten, die Beleuchtung zu dimmen und Benachrichtigungen (z.B. von offenem Fenster, beendetem Waschgang uvm.) anzuzeigen. Für eine automatische Helligkeitsregulierung sollte noch ein Lichtmesser eingebaut sein.

\textbf{Bedienelement:}\\
-Touchscreen

\textbf{Benötigter Sensor:}\\
-Lichtsensor

\subsection{Finanztechnologie - FinTech}
Für Geld Transaktionen zu führen wird eine drahtlose Verbindungsart vorausgesetzt. Transaktionen werden Bereits erwähnte Applikationen wie Apple Pay und Google Wallet verwenden den NFC (Near Field Communication) Chip und TWINT verwendet Bluetooth. Die Kommunikationsschnittstelle ist notwendig um Zahlungsanforderungen zu erhalten und diese auch zu authorisieren. Zum auswählen von Zahlungsoptionen und bestätigen von Überweisungen ist ein Bedienelement notwendig. Wenn mehrere Kreditkarten hinterlegt wären, muss der Anwender die Möglichkeit haben, um eine davon auszuwählen für die Überweisung.

\textbf{Auswahl von benötigten Kommunikationsschnittstellen:}\\
- NFC Chip\\
- Bluetooth\\
- GSM/UMTS/LTE\\
- Wireless LAN (weniger geeignet)

\textbf{Mögliche Bedienelemente:}\\
- Touchscreen (beste Vorraussetzung)\\
- Mikrofon zur Sprachsteuerung\\
- Physische Taste\\

\section{siot.net Applikationen Anforderungen}
Technische Voraussetzung für die siot.net Applikationen werden für alle in der Bedürfnisanalyse ermittelten Klassen definiert.

\subsection{siot.net Gateway Library}
Die Bibliothek, welche smarte Geräte an die siot.net Plattform anbinden soll, wird in erster Linie nur für das Android Betriebsystem entwicklet. Für die Benutzung von Apps welche ans siot.net angeschlossen werden, müssen diese minimal die Android Tools der SDK Version 22 beherrschen (ab Android 5.0 Lollipop). Um die Daten an den MQTT Broker übermitteln zu können, braucht es ein Netzwerkmodul, die eine Verbindung ins Internet erlaubt. Die Bibliothek sollte alle verfügbaren und bekannten Sensoren selber erkennen. Die Verwendung, ob Sensoren aktiviert werden oder nicht, wird durch den Appentwickler oder der Software selber definiert.

\textbf{Auswahl von benötigten Kommunikationsschnittstellen:}\\
- NFC Chip\\
- Bluetooth\\
- GSM/UMTS/LTE\\
- Wireless LAN

\textbf{Betriebssystem:}
- ab Android 4.4
- ab SDK Tools Version 22 (Android Tools)

\subsection{siot.net Dashboard App}
Um auch Sensordaten von der siot.net Plattform darzustellen eignet sich eine Dashboard App. Auch diese wird vorraussichtlich nur für Android Geräte implementiert. Für die Darstellung einer derartigen digitalen Instrumententafel eignet sich ein Display, bevorzugt ein Touchscreen. Ein Berührbildschirm erlaubt es die gewünschten Anzeigen bequem einzublenden. Eine weitere Voraussetzung ist die Vernetzung. Das Gerät muss einen Zugang zum Internet herstellen können. Nur durch eine erfolgreiche Anmeldung an den siot.net MQTT Broker, können die Informationen empfangen werden.

\textbf{Auswahl von benötigten Kommunikationsschnittstellen:}\\
- NFC Chip\\
- Bluetooth\\
- GSM/UMTS/LTE\\
- Wireless LAN

\textbf{Bevorzugtes Bedienelement:}\\
- Touchscreen\\

\textbf{Betriebssystem:}
- ab Android 4.4
- ab SDK Tools Version 22 (Android Tools)

\subsection{Herzfrequenz Überwachung}

\subsection{Steuerung von Modellen}
\newpage
