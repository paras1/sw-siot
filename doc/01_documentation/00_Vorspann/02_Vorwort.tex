\chapter*{Kontaktpersonen}

\textbf{Autor}\\

\author\\
Kirchweg 54\\
3324 Hindelbank\\

E-Mail: sathesh.paramasamy@students.bfh.ch\\

\vspace{1.5cm}

\textbf{Betreuender Dozent}\\

Dr. Andreas Danuser\\
Berner Fachhochschule\\
Technik und Informatik\\
Höheweg 80\\
2502 Biel\\

E-Mail: andreas.danuser@bfh.ch\\

\vspace{1.5cm}

\textbf{Beurteilender Experte}\\

Armin Blum\\
Burgunderweg 58\\
2502 Biel\\

E-Mail: armin.blum@bluewin.ch\\
\newpage
\chapter*{Management Summary}

An der Berner Fachhochschule konzipiert und entwickelt die Fachgruppe SIOT des Institutes RISIS (Research Institute for Security in the Information Society) mit Industriepartnern (AppModule und NetModule) die siot.net-Plattform, welche Sensoren und Aktoren mit Internet-of-Things-Anwendungen verbindet. Diese Plattform soll informatikfremde Personen und Unternehmen das Internet of Things näherbringen. Das Ziel ist Dinge zu vernetzen, welche miteinander kommunizieren sollen. Diese Umgebung befindet sich in der Erst-Implementationsphase und bestrebt für so viele "`Things"' wie möglich den Zugang zu ermöglichen.

Einige dieser Dinge, die im Internet der Dinge eine grosse Rolle spielen, sind Smartwatches und Smartphones. Diese integrieren eine grosse Anzahl von Sensoren und Aktoren. Mit der effizienten Anbindung dieser Geräte an die siot.net-Plattform, können vielseitige Messdaten versendet und einkommende Informationen genutzt werden.\\
Es sollen Bedürfnisse, Geräte und Anwendungsfälle für Smartwatches evaluiert und ausgewählt werden und diese mit einer generischen Architektur an die siot.net-Plattform angebunden werden.

Die vorliegende Bachelorthesis beschreibt einen Ansatz, wie Smartwatches und Smartphones ans siot.net angebunden wird. Die in dieser Arbeit spezifizierte Netzwerk-Architektur zeigt, wie die Kommunikation behandelt wird. In der Konzeption liegt der Schwerpunkt eine generische Bibliothek zu designen, welche im Android Umfeld wiederverwendbar ist. Diese übernimmt die Verantwortung für das korrekte Management der Verbindungen und des Nachrichtentransfers zwischen den mobilen Geräten und der siot.net-Plattform. Anhand diesem Ansatz wird Applikationsentwicklern die Anbindung ans siot.net vereinfacht.

Um die Bibliothek zu verifizieren, wird eine Lösung in Form einer Android App entworfen und umgesetzt. Mit Hilfe der entwickelten Applikation, sollen alle Messwerte der eingebauten, standardisierten Sensoren, von Android Smartphones und Android Wear Smartwatches, publiziert werden.
Für einen gewinnbringenden Einsatz der generischen Bibliothek zu ermöglichen, sind noch weitere Aspekte zu identifizieren und nachfolgend zu implementiert. Mit konsequenter Verfolgung der spezifizierten Anforderungen, Konzepte und Nacharbeiten, soll diese Bibliothek in einer Release-Version entstehen. Diese wird Android-Applikationsentwicklern unterstützen, ihre Apps mit wenig Aufwand ans siot.net anzubinden.
