\chapter{Bedürfnisanalyse}
\section{Smartwatch Applikationen}
\subsection{Überwachung}
\textbf{Gesundheit}\\
Sturz erkennen:\\
Es wird ein Alarm ausgelöst, wenn nicht innerhalb von ca. 30s Bestätigung erfolgt und keine Bewegungen stattfinden

Puls überwachen:\\
Auch hier kann Alarmiert werden, wenn keine der Puls zu niedrig/hoch ist und vom Träger keine Aktionen erfolgen\\

\textbf{Alarming}\\
Spital: \\
Pflegeperson kann Patientenalarm direkt auf die Smartwatch erhalten

Haushalt: \\
Geräte im Haushalt können überwacht werden. Dies hilft Gefahren abzuwenden sowie Zeit zu optimieren.
z.B. Wenn eine Herdplatt noch läuft wird ein Alarm ausgelöst.
Oder wenn die Waschmaschine ihren Waschgang beendet hat kann der Träger dirket benachrichtigt werden.

\textbf{Sport}\\
Bewegungen: \\
Die getätigten Bewegungen beim Sport aufzeichnen mit den vorhandenen Sensoren.
Körper-Belastung messen wie z.B. Beschleunigung, Geschwindigkeit, Stärke usw.

\subsection{Fernbedienung}
Smart Home: \\
Das Fernbedienen von Geräte im Haushalt dürfte eine der interessantesten Anwendungsbereiche sein.
Da sind unbegrenzte Möglichkeiten vorhanden. Man kann das Licht steuern, nicht nur ein und ausschalten sondern auch gleich dimmen und timen.
Multimediageräte wie der TV oder die Set-Top-Box können aus dem Handgelenk gesteuert werden.
Mit der haptischen Rückmeldung, könnten sogar sehbehinderte Menschen davon profitieren.

\subsection{Umgebung}
Reminder:\\
Wen die Uhr aus dem Sichtbarkeitsumfeld des Smartphones gelangt, kann der Träger informiert werden.\\
Durch das erreichen einer Geofencing Zone können Erinnerungen ausgelöst werden.

Radar:\\
Es können Leute in der nähe ermittelt werden. Dies kann zu verschiedenen Zwecken eingesetzt werden.\\
Datingportale können diese Funktionen interessant einsetzen. Potenzielle Datingpartner können gleich mit Foto auf der Smartwatch angezeigt werden.
Die daraus resultierenden Kontakt Möglichkeiten wären, direkt anchatten, auf sich aufmerksam machen, ignorieren uvm.

\subsection{Navigation}
Indoornavigation:\\
In Zusammenarbeit mit Beacons/Eddystones und/oder Access Points können jeweils die Standorte von den Träger der Smartwatch ermittelt werden.
Dies ermöglicht Grossfirmen, die Mitarbeiter sich zu finden ohne direkten Kontakt zu haben.
Auch das Problem mit den Shared-Desk Arbeitsplätzen, dass diese meist besetzt sind man nicht weiss wo der nächste freie Platz ist, kann gelöst werden.
Mit der Smartwatch kann man sich bei einem freien Arbeitsplatz sich anmelden und diesen reservieren, nur durch erreichen des Schreibtisches.

\subsection{Authentifikation}
Türen:\\
Um aller Art Türen zu entriegeln können Smartwatches gebraucht werden.

Zugangskontrollen:\\
Die Smartwatch hat das Potenzial Personalausweise zu ersetzen. Zeitgleich kann es auch zu Zeiterfassung genutzt werden.
Der Mitarbeiter muss nicht mehr an die Zeiterfassungsleser, eintretten und austretten der Arbeitsumgebung kann automatisch erkannt und erfasst werden.

\subsection{Finanztechnologie - FinTech}
Zahlungen:\\
Die Möglichkeit nur mit der Uhr zu zahlen besteht. Es gibt bereits Lösungen welche mit Smartphones funktionieren {(Twint/Apple Pay/Google Wallet)}.
Diese Funktionen können auch auf die Smartwatch erweitert werden.

\section{Smartwatch Applikationen für siot.net}

Die siot.net Plattform bietet sich bestens als Kommunikationsschnittstelle an für die Applikationen, welche im vorherigen Abschnitt ermittelt wurden.
Somit sollte jede dieser Applikationen problemlos mit siot.net verknüpft werden können.

Um Verknüpfungen verschiedenster Applikationen mit einer Plattform zu erstellen sollte es eine generische Biblithek geben.
Diese sollte eine einfache Schnittstelle von der Applikation zu siot.net Plattform implementieren.

\subsection{siot.net Android Gateway Library}

TODO

\subsection{siot.net Dashboard App}

TODO
