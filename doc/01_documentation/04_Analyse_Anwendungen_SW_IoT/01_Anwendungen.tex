\chapter{Bedürfnisanalyse}

Im Kapitel der Bedürfnisanalyse werden Anwendungsfälle ermittelt welche mit Smartwatches abgedeckt werden können. Zusätzlich wird eine Bedarfsanalyse durchgeführt für Applikationen welche in Verbindung zur siot.net Plattform stehen.
\section{Smartwatch Applikationen}
\subsection{Gesundheit}
Im Gesundheitssektor gibt es einige Anwendungsfälle welche mit einer smarten Uhr abgedeckt werden können.

Eine Smartwatch bietet die Möglichkeit sich zu überwachen. Da die Computeruhr mit vielen Sensoren, wie z.B. Bewegungsensor oder Herzfrequenzmesser, ausgerüstet ist, hat sie die Möglichkeit den Träger sehr genau zu analysieren.

Bei einem Sturz des Benutzers kann ein Alarm ausgelöst werden. Dieser würde in erster Instanz eine positive Gesundheitsbericht des Anwenders verlangen. Diese Bestätigung sollte in einem definierten Zeitrahmen statt finden. Falls dies nicht ausgeführt wird und die Uhr keine Bewegung registriert, kann ein Alarm an eine Vertrauensperson oder gar ein Notruf ausgelöst werden. Dieser Notruf kann wichtigen Daten angereichert werden, wie z.B. Pulsdaten und die GPS Koordinaten. Statt des Sturzes kann hier der Auslöser des Alarmes auch ein zu tiefer oder gar kein Puls sein.

Ein weiterer nützlicher Use-Case ist, Alarme von Patienten im Spital. Hier kann das Pflegepersonal mit Smartwatches, Patientenalarme erhalten. Die Alarme sollten möglichst nur empfangen werden, wenn der Patient in der nähe des Patienten sich befindet. Wenn mehrere Pfleger/innen benachrichtigt werden, kann eine Pflegeperson den Alarm bestätigen und die Verantwortung für den Patient übernehmen, so können Doppelspurigkeiten vermieden werden.

\subsection{Smart Home}
Für die Fernbedienung von Geräten im Haus oder Wohnung eignet sich die Smartwatch gut. Mit eingebauten Touchscreen und Vibrationsmotor, haben die kleinen Handgelenkrechner die Möglichkeit Informationen visuell wie taktil an die Person zu bringen.

Geräte im Haushalt können überwacht werden. Dies hilft Gefahren abzuwenden. Wenn eine Herdplatt noch läuft kann ein Alarm ausgelöst werden und es kann gleich mit der Uhr reagiert werden und die Platte ausschalten.

Für jeden einen Mehrwert gibt die Funktion Licht vom Handgelenk zu bedienen. Das Zimmer beleuchten ohne zum Lichtschalter gehen zu müssen. Dimmen mit dem Touchscreen und terminierte Lichtsteuerung.

Ein weiterer Anwendungsfall ist die Waschmaschine. Die Restzeit des Waschgangs kann auf den Bildschirm angezeigt werden und wenn er beendet ist, wird der Träger mit einem Vibrationsimpuls notifiziert.

Desweiteren ist das Fernbedienen von allen Multimediageräten vom Handgelenk sehr praktisch. Es genügt eine Uhr und braucht nicht mehr viele verschiedene proprietäre Steuerungen. Dies wird heute bereits mit Smartphone Apps praktiziert. Mit Sprachsteuerung können Personen mit eingeschränkter Sehkraft die Uhr verwenden.

\subsection{Sport}
Heute werden Smartwatches hauptsächlich als Fitnesstracker verwendet.\footnote{Quelle: Studienband Smartwatch Umfrage eResult, Stand: 15.11.2015} Das Praktische an den Uhren unter den Wearables ist, dass diese nicht nur für zum Sport treiben gekauft werden muss. Hier erhält der Endkunde ein Gerät für den Alltag und die Freizeit.

Im Sportbereich kann mit den vorhandenen Sensoren viele verschiedene Werte ermittelt und analysiert werden. Mit den nötigen Voreinstellungen, wie Körpermasse, Schrittlänge, Alter und Geschlecht, ist es möglich Bewegungsdaten genau aufzuzeichnen. Mit den Daten können für den Anwender interessante Informationen berechnet werden. Für Hobbysportler meist relevante Berechnungen sind Zeit, Schritte, Geschwindigkeit und Kalorienverbrauch. Für erfahrene Sportler verbessern ihre Fähigkeiten durch betrachten von Auswertungen der Körperbelastungen, z.B. Beschleunigung, Stärke, Drehmoment und weitere.

\subsection{Umgebung}
Applikationen welche umgebungsorientiert arbeiten, sind geeignete Kandidaten für Smartwatches. Durch die permanente Anzeige am Handgelenk, können sich schnell ändernde Daten dauerhaft im Auge behalten werden.

Ein Anwendungsfall ist, das Smartphone zu überwachen. Die Uhr kann den Träger informieren, wenn das Sichtbarkeitsumfeld vom Mobiltelefon und des tragbaren Rechners sich nicht mehr überschneiden, was bedeuten würde die Geräte entkoppeln sich voneinander.

Die gleiche Methode bietet sich an für Personen mit Smartwatches in der nähe zu scannen. Diese Funktion kann für Partnervermittlungsapplikationen effektiv eingesetzt werden. Auf dem Touchscreen können potenzielle Datingpartner angezeigt und kontaktiert werden. In der schnelllebigen Welt sind sich schnell erschliessende Kontakte sehr willkommen.

Ein weiterer Punkt ist Geofencing. Dies kann effektiver genutzt werden, da durch den immer sichtbaren Bildschirm, die ortsrelevanten Daten  in nützlicher Zeit abgerufen werden können.

TODO

\subsection{Navigation}
Indoornavigation:\\
In Zusammenarbeit mit Beacons/Eddystones und/oder Access Points können jeweils die Standorte von den Träger der Smartwatch ermittelt werden.
Dies ermöglicht Grossfirmen, die Mitarbeiter sich zu finden ohne direkten Kontakt zu haben.
Auch das Problem mit den Shared-Desk Arbeitsplätzen, dass diese meist besetzt sind man nicht weiss wo der nächste freie Platz ist, kann gelöst werden.
Mit der Smartwatch kann man sich bei einem freien Arbeitsplatz sich anmelden und diesen reservieren, nur durch erreichen des Schreibtisches.

\subsection{Authentifikation}
Türen:\\
Um aller Art Türen zu entriegeln können Smartwatches gebraucht werden.

Zugangskontrollen:\\
Die Smartwatch hat das Potenzial Personalausweise zu ersetzen. Zeitgleich kann es auch zu Zeiterfassung genutzt werden.
Der Mitarbeiter muss nicht mehr an die Zeiterfassungsleser, eintretten und austretten der Arbeitsumgebung kann automatisch erkannt und erfasst werden.

\subsection{Finanztechnologie - FinTech}
Zahlungen:\\
Die Möglichkeit nur mit der Uhr zu zahlen besteht. Es gibt bereits Lösungen welche mit Smartphones funktionieren {(Twint/Apple Pay/Google Wallet)}.
Diese Funktionen können auch auf die Smartwatch erweitert werden.

\section{Smartwatch Applikationen für siot.net}

Die siot.net Plattform bietet sich bestens als Kommunikationsschnittstelle an für die Applikationen, welche im vorherigen Abschnitt ermittelt wurden.
Somit sollte jede dieser Applikationen problemlos mit siot.net verknüpft werden können.

Um Verknüpfungen verschiedenster Applikationen mit einer Plattform zu erstellen sollte es eine generische Biblithek geben.
Diese sollte eine einfache Schnittstelle von der Applikation zu siot.net Plattform implementieren.

\subsection{siot.net Android Gateway Library}

TODO

\subsection{siot.net Dashboard App}

TODO
