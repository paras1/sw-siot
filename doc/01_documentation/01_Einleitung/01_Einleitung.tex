\chapter{Einleitung}
Dieses Dokument dient dazu die Marktanalyse zu erläutern von Internet of Things und Smartwatches.
Die in diesem Dokument behandelte Marktanaylse besteht aus der Marktsegementierung und der Bedürfnisanalyse.
Mit der Bedürfnisanalyse werden Applikationen für Smartwatches ermittelt, die im oder für siot.net angewendet werden können.
Für die gefundenen Applikationen resultieren noch technische Anforderungen an Smartwatches.

\section{Internet of Things (IoT)}
Das Internet der Dinge (Internet of Things / IoT) ist ein Gebilde, bei dem Objekte, Tiere oder Menschen mit einem einzigartigen Identifikator ausgestattet sind. Weiterhin ist damit die Möglichkeit verbunden, Daten über ein Netzwerk ohne Interaktionen Mensch-zu-Mensch oder Mensch-zu-Computer zu übertragen. Das Internet der Dinge hat sich aus der Konvergenz der drahtlosen (wireless) Technologie, MEMS (Micro-Electromechanical Systems) und dem Internet entwickelt.

Ein Ding im Internet der Dinge kann zum Beispiel eine Person mit einem Herzschrittmacher, ein Nutztier auf einem Bauernhof mit einem Biochip-Transponder oder ein Automobil mit eingebauten Sensoren sein. Letzteres könnte eine Warnung auslösen, wenn der Reifendruck zu niedrig ist. Im Prinzip ist jedes vom Menschen geschaffene Objekt ein Kandidat, das sich mit einer IP-Adresse ausstatten lässt und Daten via Netzwerk übertragen kann. Bisher wurde das Internet der Dinge am häufigsten mit M2M-Kommunikation (Maschine-zu-Maschine) bei der Fertigung, sowie der Strom-, Gas- und Öl-Versorgung in Verbindung gebracht. Sind Produkte mit M2M-Kommunikation ausgestattet, werden sie häufig als intelligent oder smart bezeichnet.\footnote{Quelle: \url{http://www.searchnetworking.de/definition/Internet-der-Dinge-Internet-of-Things-IoT}, Stand: 23.10.2015}


\section{Smartwatches}
Smartwatches sind kompakte Computersysteme, welche vom Benutzers am Handgelenk getragen werden kann. Diese sind meist mit einer oder mehreren drahtlos Technologie und verschiedenen Sensoren (Bewegungssensor, Lichtsensor, Herzfrequenzmesser) Aktoren (Bildschrim, Vibrationsmotor) ausgerüstet.
Diese Uhren unterstützen den Träger beim alltäglichen Leben. Gehören zur Gruppe der Wearables und damit zu einem essentiellen Bereich des IoT.

Mit einer Smartwatch können viele verschiedene Funktionalitäten mit einem Gerät abgedeckt werden.

\subsection{Beispiele}
\begin{tabular}{ll}
Pulsmessung: &	Der Träger hat seinen Puls immer unter kontrolle \\
Bewegungen:	& Ob eine Person sich genug bewegt, kann gemessen werden \\
Fitness: & Genaue Bewegungen können registriert werden \\
Informationen: & Der Träger kann Informationen empfangen welche auf seinem Smartphone ersichtlich sind
\end{tabular}

\subsection{Fachbereich Informatik}
Smartwatches gehören in den Bereich der Wearables. Und Wearables ist ein fachübergreifendes Gebiet der Informatik, einige Fachgebiete:

Ubiquitous Computing, die Rechnerallgegenwärtig \\
Pervasive Computing, die Vernetzung von Alltagsgegenständen \\
Mobile Computing, mobile Mensch zu Maschinen Kommunikation \\
M2M, Machine-to-Machine, Informationsaustausch zwischen Zielgeräten \\
IoT, Internet of Things, dass auf den vorhergehenden Fachbereichen basiert

\section{siot.net}
