%
% Main document
% ===========================================================================
% This is part of the document "Smartwatches in siot.net".
% Authors: paras1
%

% Document informations
%---------------------------------------------------------------------------
\def \module		{Bachelor Thesis}					% Module name
\def \title			{Smartwatches in siot.net}		% Title
\def \version		{X1.0}
\def \author		{Sathesh Paramasamy}
\def \logo			{BFH_Logo_C_de_fr_en_100_4CU}	% choose the correct logo in
													% the folder Bilder/BFH_Logo
%-----------------------------------------------------------------%----------

%---------------------------------------------------------------------------
\documentclass[
	a4paper,				% paper format
	10pt,					% fontsize
%	twoside,				% double-sided
	oneside,				% one-sided
	openright,				% begin new chapter on right side
	notitlepage,			% use no standard title page
	parskip=half,			% set paragraph skip to half of a line
]{scrreprt}					% KOMA-script report
%---------------------------------------------------------------------------

\raggedbottom
\KOMAoptions{cleardoublepage=plain}	% Add header and footer on blank pages


% Load Standard Packages:
%---------------------------------------------------------------------------
\usepackage[standard-baselineskips]{cmbright}

\usepackage[ngerman]{babel}		% german hyphenation
%\usepackage[latin1]{inputenc}  % Unix/Linux - load extended character set (ISO 8859-1)
%\usepackage[ansinew]{inputenc}  % Windows - load extended character set (ISO 8859-1)
\usepackage[utf8]{inputenc}    % UTF-8 - load extended charater set
\usepackage[T1]{fontenc}		% hyphenation of words with ä,ö and ü
\usepackage{textcomp}			% additional symbols
\usepackage{ae}					% better resolution of Type1-Fonts
\usepackage{fancyhdr}			% simple manipulation of header and footer
\usepackage{graphicx}			% integration of images
\usepackage{float}				% floating objects
\usepackage{caption}			% for captions of figures and tables
\usepackage{booktabs}			% package for nicer tables
\usepackage{tocvsec2}			% provides means of controlling the sectional numbering
\usepackage{rotating}			% rotating tables and other objects
\usepackage{pdflscape}			% change single pages landscape
\usepackage{tabularx}			% create nice tables
\usepackage{pdfpages}			% insert full pdf pages
\usepackage{nameref}			% reference by name, not by chapter number
\usepackage{dirtree}			% create directory trees
\usepackage{listings}			% include source code
\usepackage{epstopdf}			% convert eps graphics to pdf
%---------------------------------------------------------------------------

% Load Math Packages
%---------------------------------------------------------------------------
\usepackage{amsmath}			% various features to facilitate writing math formulas
\usepackage{amsthm}				% enhanced version of latex's newtheorem
\usepackage{amsfonts}			% set of miscellaneous TeX fonts that augment the standard CM
\usepackage{amssymb}			% mathematical special characters
\usepackage{exscale}			% mathematical size corresponds to textsize
%---------------------------------------------------------------------------

% Package to facilitate placement of boxes at absolute positions
%---------------------------------------------------------------------------
\usepackage[absolute]{textpos}
\setlength{\TPHorizModule}{1mm}
\setlength{\TPVertModule}{1mm}
%---------------------------------------------------------------------------

% Definition of Colors
%---------------------------------------------------------------------------
\RequirePackage{color}							% Color (not xcolor!)
\definecolor{linkblue}{rgb}{0,0,0.8}				% Standard
\definecolor{darkblue}{rgb}{0,0.08,0.45}			% Dark blue
\definecolor{brickred}{cmyk}{0,0.89,0.94,0.28}	% Brickred
\definecolor{bfhred}{rgb}{0.776,0,0.066}			% Red
% specific colors
\definecolor{titlecolor}{rgb}{0,0.08,0.45}		% Color used for the title
%\definecolor{linkcolor}{rgb}{0,0,0.8}			% Blue for the web- and cd-version!
\definecolor{linkcolor}{rgb}{0,0,0}				% Black for the print-version!
\definecolor{code_bg}{gray}{0.8}					% Source Code Background
%---------------------------------------------------------------------------

% Hyperref Package (Create links in a pdf)
%---------------------------------------------------------------------------
\usepackage[
	pdftex,ngerman,bookmarks,plainpages=false,pdfpagelabels,
	backref = {false},					% No index backreference
	colorlinks = {true},					% Color links in a PDF
	hypertexnames = {true},				% no failures "same page(i)"
	bookmarksopen = {true},				% opens the bar on the left side
	bookmarksopenlevel = {0},			% depth of opened bookmarks
	pdftitle = {\title},					% PDF-property
	pdfauthor = {\author},				% PDF-property
	pdfsubject = {\module},				% PDF-property
	linkcolor = {linkcolor},				% Color of Links
	citecolor = {linkcolor},				% Color of Cite-Links
	urlcolor = {linkcolor},				% Color of URLs
]{hyperref}
%---------------------------------------------------------------------------

% Set up page dimension
%---------------------------------------------------------------------------
\usepackage{geometry}
\geometry{
	a4paper,
	left=20mm,
	right=15mm,
	top=30mm,
	headheight=20mm,
	headsep=10mm,
	textheight=232mm,
	footskip=15mm
}
%---------------------------------------------------------------------------

% Makeindex Package
%---------------------------------------------------------------------------
\usepackage{makeidx}				% To produce index
\makeindex						% Index-Initialisation
%---------------------------------------------------------------------------

% Glossary Package
%---------------------------------------------------------------------------
% the glossaries package uses makeindex
% if you use TeXnicCenter do the following steps:
%  - Goto "Ausgabeprofile definieren" (ctrl + F7)
%  - Select the profile "LaTeX => PDF"
%  - Add in register "Nachbearbeitung" a new "Postprozessoren" point named Glossar
%  - Select makeindex.exe in the field "Anwendung" ( ..\MiKTeX x.x\miktex\bin\makeindex.exe )
%  - Add this [ -s "%tm.ist" -t "%tm.glg" -o "%tm.gls" "%tm.glo" ] in the field "Argumente"
%
% for futher informations go to http://ewus.de/tipp-1029.html
%---------------------------------------------------------------------------
\usepackage[nonumberlist]{glossaries}
\makeglossaries
%---------------------------------------------------------------------------

% Listings Package
%---------------------------------------------------------------------------
\lstdefinestyle{CCode}{
	showspaces=false,
	showtabs=false,
	language={[ANSI]C},
	breaklines=true,
	basicstyle={\footnotesize \ttfamily},
	backgroundcolor=\color{code_bg},
	%frame=single,
	tab=\rightarrowfill,
	captionpos=b
}

\lstdefinestyle{CppCode}{
	showspaces=false,
	showtabs=false,
	language={[ISO]C++},
	breaklines=true,
	basicstyle={\footnotesize \ttfamily},
	backgroundcolor=\color{code_bg},
	%frame=single,
	tab=\rightarrowfill,
	captionpos=b
}

\lstdefinestyle{JavaCode}{
	showspaces=false,
	showtabs=false,
	language={Java},
	breaklines=true,
	basicstyle={\footnotesize \ttfamily},
	backgroundcolor=\color{code_bg},
	%frame=single,
	tab=\rightarrowfill,
	captionpos=b
}
%---------------------------------------------------------------------------

% Intro:
%---------------------------------------------------------------------------
\begin{document}					% Start Document
\settocdepth{section}			% Set depth of toc
\pagenumbering{Roman}
%---------------------------------------------------------------------------

% Set up header and footer
%---------------------------------------------------------------------------
\fancyhf{}						% clean all fields
\fancypagestyle{plain}{			% new definition of plain style

% Use this for double-sided:
%	\fancyfoot[OL,ER]{\footnotesize				% footer left part -->	version
%		V\version \\
%		\author \\
%		\today
%	}
%	\fancyfoot[OR,EL]{\footnotesize \thepage}	% footer right part --> page number
%	\fancyhead[C]{\module}						% header right part --> module name
%	\fancyhead[OR,EL]{\footnotesize \leftmark}	% footer left part -->	chapter
%	\fancyhead[OL,ER]{							% header left part --> BFH logo
%		\begin{textblock}{0}[0,0](29,9)
%			\includegraphics[scale=1.0]{Bilder/\logo}
%		\end{textblock}
%	}

% Use this for one-sided:
	\fancyfoot[L]{\footnotesize					% footer left part -->	version
		\version \\
		\author \\
		\today
	}
	\fancyfoot[R]{\footnotesize \thepage}		% footer right part --> page number
	\fancyhead[C]{\module}						% header right part --> module name
	\fancyhead[R]{\footnotesize \leftmark}		% footer left part -->	chapter
	\fancyhead[L]{								% header left part --> BFH logo
		\begin{textblock}{0}[0,0](20,9)
			\includegraphics[scale=0.9]{Bilder/BFH_Logo/\logo}
		\end{textblock}
	}
}

\renewcommand{\chaptermark}[1]{\markboth{\thechapter.  #1}{}}
\renewcommand{\headrulewidth}{0pt}				% no header stripline
\renewcommand{\footrulewidth}{0pt}				% no bottom stripline

\pagestyle{plain}
%---------------------------------------------------------------------------


% Title Page and Abstract
%---------------------------------------------------------------------------
% ===========================================================================
% This is part of the document "Smartwatches in siot.net".
% Authors: paras1
%

\begin{titlepage}


% BFH-Logo absolute placed at (29,10) on A4
% Actually not a realy satisfactory solution but working.
%---------------------------------------------------------------------------
\begin{textblock}{0}[0,0](20,10)
	\includegraphics[scale=1.0]{98_Bilder/99_BFH_Logo/\logo}
\end{textblock}

% Titel / Untertitel / Autor:
%---------------------------------------------------------------------------
\begin{flushleft}

\vspace*{2cm}

\fontsize{24pt}{28pt}\selectfont
\textbf{\module} \\
\textbf{\title} \\
\fontsize{12pt}{15pt}\selectfont
\vspace{0.5cm}
\titleborder \\
\vspace{10cm}
\textbf{Autor}\\
\author					\\
\vspace{1cm}
\textbf{Betreuender Dozent}\\
Dr. Andreas Danuser\\
\vspace{3.5cm}
Version \version					\\
\today					\\
\end{flushleft}


\end{titlepage}

%
% ===========================================================================
% EOF
%

\cleardoubleemptypage
\setcounter{page}{1}
\chapter*{Kontaktpersonen}

\textbf{Autor}\\

\author\\
Kirchweg 54\\
3324 Hindelbank\\

E-Mail: sathesh.paramasamy@students.bfh.ch\\

\vspace{1.5cm}

\textbf{Betreuender Dozent}\\

Dr. Andreas Danuser\\
Berner Fachhochschule\\
Technik und Informatik\\
Höheweg 80\\
2502 Biel\\

E-Mail: andreas.danuser@bfh.ch\\

\vspace{1.5cm}

\textbf{Beurteilender Experte}\\

Armin Blum\\
Burgunderweg 58\\
2502 Biel\\

E-Mail: armin.blum@bluewin.ch\\
\newpage

\chapter*{Management Summary}

Hier steht das Managment Summary

%---------------------------------------------------------------------------

% Table of contents and listings
%---------------------------------------------------------------------------
\tableofcontents
\cleardoublepage
%---------------------------------------------------------------------------

\pagenumbering{arabic}
\settocdepth{subsection}		% Set depth of toc

% Main part - Part I
%---------------------------------------------------------------------------
\onecolumn
\chapter{Einleitung}
\section{Ausgangslage}
Die Fachgruppe SIOT des Instituts RISIS der BFH konzipiert und entwickelt zusammen mit Industriepartnern (AppModule und NetModule) die Plattform siot.net, welche Sensoren und Aktoren weltweit mit \gls{IoT}-Anwendungen verbindet. Smartwaches, welche eine rasante Marktakzeptanz geniessen, spielen eine grosse Rolle im Bereich \gls{IoT}, denn sie integrieren eine Anzahl von Sensoren und können am Handgelenk Informationen anzeigen. Betreffend Funktionalität gibt es eine gewisse Spannweite bei den Smartwatches, was deren mögliche Einsatzgebiete schliesslich definiert.

\section{Problemstellung}
Die Projektarbeit 2 erlaubte Android Smartwatch zu analysieren. Diese Erkenntnisse sollen genutzt und mit einer praktischen Umsetzung konkretisiert werden.
Dabei sollen folgende Themen genauer betrachtet werden: \\
- Welche Anwendungsklassen kann man für Smartwatches erkennen? \\
- Wie werden Smartwaches am weltweiten Internet angebunden? \\
- Welche GUI-Elemente werden bereitgestellt? \\
- Welche Sensoren und Aktoren stehen zur Verfügung? \\

Die Bachelorarbeit beinhaltet eine Markt- und Bedürfnisanalyse, welche die Marktsegmente und die Bedürfnisse aus Sicht \gls{IoT} für Smartwatch aufzeigen. Für die identifizierten Anwendungen werden Smartwatches evaluiert.\\
Als weitere Aufgabe wird eine generische System-Architektur definiert, mit welcher Software für Smartwatches für \gls{IoT}-Anwendungen im siot.net Umfeld realisiert werden kann. \\
In einem formalen Teil werden die Anforderung bzw. technischen Anforderung, einer bestimmten Smartwatch an eine Anwendung gestellt, untersucht und aufgezeigt. Hierbei sollen auch Genauigkeiten und Zuverlässigkeit betrachtet werden. \\
Es wird ein Softwaredesign erstellt mit welchem zwei bis drei konkrete Anwendungen implementiert werden könnten. \\
Daraus wird mindestens eine konkrete Anwendung umgesetzt. Zur Realisierung wird eine Dokumentation erstellt welche von Informatik-Ingenieuren gelesen wird. \\
Schlussendlich werden in diesem Dokument alle Ergebnisse berichtet.

\section{Zielsetzung}
Mit einer Bedürfnisanalyse sollen Anwendungsfälle für Smartwatches erarbeitet werden. Mit den entdeckten Use-Cases werden aktuelle Smartwatches evaluiert und mindestens eine wird genauer betrachtet. Um eine geeignete Plattform für die Softwareentwicklung der gewählten Uhr aufzubauen, wird eine Entwicklungsumgebung eruiert.

Die erstellten Grundlagen helfen eine generische Softwarebibliothek zu erstellen. Mit diesem Stück Software wird ermöglicht, Smartwatches und Smartphones schnell an die siot.net-Plattform anzubinden. Entwickler von Apps erleichtert dies die Arbeit, denn für die Verbindung an das \gls{IoT}-System kann die Bibliothek verwendet werden. Bei der generischen Anbindung wird in erster Linie das Hauptaugenmerk auf Sensordaten, sowie die Verbindung von Smartwatches gelegt. Zusätzlich wird Programmierern ein Entwicklerhandbuch bereitgestellt. Dieses erläutert die Möglichkeiten (JavaDoc) und beschreibt ein kleines Tutorial.

Einige Funktionen werden mit einer Applikation gezeigt, welche die siot.net Anbindungsbibliothek integriert.

\section{Abgrenzung}
Smartwatches im Allgemeinen gibt es von vielen verschiedenen Anbietern. In dieser Arbeit werden aller Art Smartwatches analysiert. Der Schwerpunkt für das Softwaredesign und die Implementationen liegt auf Smartwatches, die mit dem Betriebssystem Android ausgeliefert werden. Diese Abgrenzung findet statt, um die Entwicklung auf ein offenes System (Open Source) zu beschränken. Weiterhin wird die Stabilität und Sicherheit nicht genauer betrachtet, was im Rahmen dieser Arbeit nicht abzudecken wäre.

\section{Projektmanagement}
Für die Organisation der Arbeit ist ein Zeitplan erstellt und eine Prozesssteuerung verwendet worden. Als Prozesssteuerung diente Kanban. Da dies eine Einzelarbeit war, hielt die schlanken Zeit- und Prozessplanungsmethoden den Overhead in Grenzen. Als Dokumenten- und Quelltextablage, diente die Cloud Plattform github.com, diese verwendet als Versionskontrollsystem Git. Zu finden sind die Daten auf folgendem Verzeichnis: \url{https://github.com/paras1/sw-siot}.

\subsection{Zeitplan}
Für die Zeitplanung kam ein tabellarischer Zeitplan zum Einsatz. Der Plan ist im Anhang zu betrachten.

\subsection{Prozesssteuerung - Kanban}
Das Kanban Modell kommt ursprünglich aus Japan und heisst Signalkarte. Der Begriff Signalkarte, weil auf einer Tafel der Fortschritt des Projektes sichtbar ist. Entwickelt wurde das System durch den Toyota Konzern, welches für die Fertigung ihrer Produkte dienen soll. David J. Anderson (\url{www.djaa.com}) hat das Modell im Jahre 2007 für die Informationstechnologie adaptiert.
Bei dieser Prozesssteuerungsart wird eine Prozesskette definiert und dann Aufgaben, welche Tickets genannt werden, ins Backlog erfasst. Das Kanbanboard besteht aus den Spalten der Prozesskette und dem Ticketstatus. Zur Bearbeitung dieser Bachelorthesis sind folgende Prozessspalten definiert worden: Backlog, Bereit (Ready), In Arbeit (In Progress), Erledigt (Done). Die Backlog Tickets sind dem tabellarischen Zeitplan zu entnehmen.
Typischerweise wird bei einem Kanbanprojekt eine maximale Anzahl an Tickets zur gleichzeitigen Bearbeitung definiert. Bei dieser Einzelarbeit, wurde darauf verzichtet, um den Arbeitsfluss nicht zu hindern.
Das Kanbanboard wurde nicht physisch geführt, sondern mit einer Webapplikation\footnote{Kanbanboard: \url{https://waffle.io/paras1/sw-siot}}. Diese Applikation bildet den Backlog aus der Versionisierungsablage\footnote{GitHub Backlog: \url{https://github.com/paras1/sw-siot/issues}} in die definierte Prozesstafel ab. Nach beenden eines Tickets werden diese nach fünf Tagen aus dem Prozesssteuerungssystem entfernt. Dies haltet die Übersichtlichkeit hoch. In der Versionisierungsablage (GitHub Issues) werden diese, mit dem aktuellen Status, dauerhaft gespeichert.


% Main part - Part II
%---------------------------------------------------------------------------
\onecolumn
\chapter{Marktsegmente}
Im Kapitel Marktsegmente werden die aktuellen Bereiche von Internet of Things, Smartwatches und Smartwatches im Internet of Things aufgezeigt. Es werden Themen aufgelistet und Bereich davon aufgezeigt. Hierfür ist keine strategische Marktsegmentierung durchgeführt worden, dadurch ist es keine abschliessende Auflistung. Für die Evaluation der Marktsegmente wurde der Bericht, The Internet of Things:
Mapping the value beyond the hype von McKinsey Global Institute als Referenz verwendet (vgl. \cite{mk:iot}).

\section{Marktsegmente im Internet of Things}
\begin{tabbing}
xxxxxxxxxxxxxxxxxxxx\=xxxxxxxxxxxxxxxxxxxxxxxx	\kill
Mensch:          \>  Blutdruck, Puls, Bewegungen, Schlafüberwachung, \\\>Körperanalyse {(z.B. Gewicht, Fettanteil, Wasseranteil usw.)} \\
Natur:			     \>  Erdplattenbewegung, Wasserspiegel Überwachung, Temperatur, Wind, Licht, Luft \\
Industrie:  		 \>  Maschinensteuerung, automatisierte Roboter, Lagerüberwachung \\
Heimautomation:	 \>  Nutzung und Überwachung von Haushaltsgeräte, Steuerung, Fernbedienungen \\
Automobil: 		   \>  Telemetrie, Geografische Strecke, Fahrverhalten, Nutzungsverhalten, Verkehrsbericht \\
Städte{/}Verkehr:\>  Touristisches Informationen, Dynamische Strassen, Verkehrsregulierung, Navigation, Lageberichte \\
Detailhandel:		 \>  Produktebezeichnung, Kasse, Geldüberweisung, Geldbörse
\end{tabbing}
Der Auflistung zu entnehmen, kommt das Internet der Dinge in sehr vielen verschiedenen Marktsegmenten zum tragen. Es hat noch grosses unausgeschöpftes Potential den Menschen zu unterstützen um seine Aufgaben zu erleichtern.
\subsection{Mensch}
Der Mensch ist ein wichtiges Marktsegment. Hierbei können Sensoren aller Art den Menschen analysieren. Dieser Punkt wird bei der Marktsegmentanalyse von Smartwatches und Smartwatches im Internet of Things genauer betrachtet.

\subsection{Natur}
Viele verschiedene Anwendungsfälle gibt es auch in der Natur. Es können Sensoren eingesetzt werden, um Temperaturen, Luftdruck, Luftfeuchtigkeit oder Windstärke zu messen. Mit Kombinationen von Sensoren, welche miteinander kommunizieren, können Frühwarnsysteme von Naturkatastrophen erschaffen werden. Dieses Segment hängt sehr nahe mit dem Marktsegment des Menschen zusammen.
%\begin{figure}[H]
%  \centering
%  \includegraphics[scale=0.85]{98_Bilder/03_Marktsegmente/erdbeben}
%  \caption[Frühwarnsystem von Erdbeben]{Wie ein Frühwarnsystem von Erdbeben funktioniert}
%  \footnotesize Quelle: \url{http://www.ingenieur.de/var/storage/images/media/ingenieur.de/bilder/funktionsweise-fruehwarnsystems-shakealert/3666615-1-ger-DE/Funktionsweise-des-Fruehwarnsystems-ShakeAlert_image_width_884.jpg}, Stand: 05.11.2015
%\end{figure}
\subsection{Industrie}
Das Internet der Dinge kommt in der Industrie soweit zum tragen, dass man von Industrie 4.0 spricht. Dies soll die vierte industrielle Revolution zum Ausdruck bringen. Die Fertigungstechnologie soll informatisiert werden. Auch die Logistik soll ihre Automatisierung erleben. Erreicht wird dies, weil Maschinen untereinander kommunizieren können. Möglichst alle Sektoren einer Fabrik sollen vernetzt sein. Das Ziel der Industrie 4.0 ist die intelligente Fabrik.
%\begin{figure}[h]
%  \centering
%  \includegraphics[scale=0.62]{98_Bilder/03_Marktsegmente/industrie4}
%  \caption[Industie 4.0 Symbolbild]{Industie 4.0}
%  \footnotesize Quelle: \url{https://www.scopevisio.com/ratgeber/wp-content/uploads/2015/11/Industrie-4.0.png}, Stand: 05.11.2015
%\end{figure}
%\newpage

\subsection{Heimautomation}
Die Heimautomation ist auch besser bekannt als Smart Home. Smart Home dient als Oberbegriff für technische Verfahren und Systeme in Wohnräumen und -häusern, in deren Mittelpunkt eine Erhöhung von Wohn- und Lebensqualität, Sicherheit und effizienter Energienutzung auf Basis vernetzter und fernsteuerbarer Geräte und Installationen sowie automatisierbarer Abläufe steht.

Unter diesen Begriff fällt sowohl die Vernetzung von Haustechnik und Haushaltsgeräten (z.B. Lampen, Jalousien, Heizung, aber auch Herd, Kühlschrank und Waschmaschine), als auch die Vernetzung von Komponenten der Unterhaltungselektronik (wie etwa die zentrale Speicherung und heimweite Nutzung von Video- und Audio-Inhalten).

Von einem Smart Home spricht man insbesondere, wenn sämtliche im Haus verwendeten Lampen, Taster und Geräte untereinander vernetzt sind, Geräte Daten speichern und eine eigene Logik abbilden können. Geräte sind teilweise auch getagged. Dies bedeutet, dass zu den Geräten im Smart Home Informationen z.B. über Hersteller, Produktnamen und Leistung hinterlegt sind. Dabei besitzt das Smart Home eine eigene Programmierschnittstelle, die (auch) via Internet angesprochen und über erweiterbare Apps gesteuert werden kann.

Eng verwandt mit diesen Verfahren und Systemen sind solche des Smart Metering, bei denen der Schwerpunkt auf dem Messen und einer intelligenten Regulierung des Energieverbrauchs liegt (vgl. \cite{wiki:smho}).
%\begin{figure}[h]
%  \centering
%  \includegraphics[scale=0.75]{98_Bilder/03_Marktsegmente/smarthome}
%  \caption[Smart Home Symbolbild]{Smart Home}
%  \footnotesize Quelle: \url{http://icon.asid.org/wp-content/uploads/2014/11/40349344_thumbnail.jpg}, Stand: 05.11.2015
%\end{figure}
%\newpage

\subsection{Automobil}
\gls{IoT} kann in vielen Bereichen der Automobilbranche eingesetzt werden. Es können wichtige Daten des Fahrzeugs ausgelesen werden, z.B. die Telemetriedaten. Diese können verwendet werden um das Fahrverhalten vom Lenker festzustellen. Des Weiteren kann, durch Nutzen der Daten, Probleme beim Auto ausgemacht und direkt Fahrer und Mechaniker alarmiert werden.
Interessant, für die Autobauer wie auch Autobesitzer, ist die Ortung der Fahrzeuge. Mit den aufgezeichneten geografischen Punkten kann analysiert werden, wie das Automobil verwendet wird und aktuelle verkehrsnahe Verkehrsberichte können genutzt werden. Die Vollendung der Vernetzung von Fahrzeugen ist das selbstfahrende Auto, welches alle nötigen Informationen empfängt, analysiert und verwendet, um das Ziel optimal zu erreichen.\\
Mercedes-Benz hat ein solches selbstfahrendes Forschungsfahrzeug entwickelt (vgl. \cite{mcbz:f015}). Die Abbildung 3.1 zeigt, wie das Fahrzeug durch die Sensoren einen Fussgänger erkennt und die Laserprojektionstechnik als Aktor verwendet, um dem Überquerenden die Fussgängerstreifen anzuzeigen.
\begin{figure}[h]
%  \centering
%  \includegraphics[scale=0.66]{98_Bilder/03_Marktsegmente/mercedesbenzf}
%  \caption[Selbstfahrendes Auto Mercedes-Benz F015 In Motion]{Das selbstfahrende Forschungsfahrzeug von Mercedes-Benz (Modell F015 In Motion)}
%  \footnotesize Quelle: \url{http://www.mercedes-benz.ch/content/media_library/f_015_luxury_in_motion_layer-gallery_1_01__710x396_01-2015.jpg}, Stand: 05.11.2015
  \centering
  \includegraphics[scale=0.66]{98_Bilder/03_Marktsegmente/mercedesbenzf2}
  \caption[Fussgängererkennung des Mercedes-Benz F 015]{Der Mercedes-Benz F 015 erkennt Fussgänger mit seinen Sensoren}
  \footnotesize Quelle: \url{http://www.mercedes-benz.ch/content/media_library/f_015_luxury_in_motion_gallery_05_715x230_01-2015.jpg}, Stand: 05.11.2015
\end{figure}
\newpage

\subsection{Städte und Verkehr}
In Städten gibt es sehr viele Möglichkeiten. In Verbindung mit dem Verkehr geht dies ins unermessliche. Ein sehr interessantes Thema ist die Touristik. Um ein Beispiel zu nennen: Beacons, welche nötige Information an ein smartes Gerät publizieren, um Daten von der Sehenswürdigkeit abzurufen. Dazu könnte auch gleich Empfehlungen in der Umgebung notifiziert werden. So würde für die meisten Reisenden der Reiseführer wegfallen.

Ein spektakuläres Projekt ist die dynamische Strasse: \textbf{Solar Roadways}. Das sind kleine, feste Platten, welche Photovoltaik-Elementen, Elektronik, verschiedenen Sensoren und \gls{LED}s integrieren (siehe Abbildung 3.2). Die Platten können wie Pflastersteine verlegt und miteinander verbunden werden. Durch die Sonneneinstrahlung sind sie permanent und umweltschonend mit Strom versorgt. Die \gls{LED}s können zentral gesteuert werden, um so die Fahrbahnmarkierungen anzuzeigen und z.B. aus zwei breiten Spuren drei schmale machen, spontane Parkflächen oder Verkehrszeichen. Die Sensoren können feststellen, wenn Tiere oder Menschen darüber laufen und die Fahrer, schon ein paar hundert Meter vorher, über die \gls{LED}s warnen. Und die Platten sind beheizbar. Dies erhöht die Verkehrssicherheit und verringert vermutlich Baustellen. Es wurde von Privatpersonen initiiert und gecrowdfunded. Das Vorhaben ist auf Indiegogo (\url{www.indiegogo.com}) im Juni 2014 deutlich überfinanziert abgeschlossen worden. In Holland hat man im Jahr 2014 begonnen, Radwege auf diese Weise zu bauen (vgl. \cite{hoco:sorw}).
\begin{figure}[h]
  \centering
  \includegraphics[scale=0.61]{98_Bilder/03_Marktsegmente/solar_roadway_00}
  \caption[Solar Roadway, dynamische Strasse]{Das Solar Roadway verändert die Strasse für die aktuelle Verkehrsituation}
  \footnotesize Quelle: \url{http://www.solarroadways.com/images/intro/Downtown%20Sandpoint%202%20-%20small.jpg}, Stand: 05.11.2015
\end{figure}
%\newpage
%\begin{figure}[H]
%  \centering
%  \includegraphics[scale=0.5]{98_Bilder/03_Marktsegmente/solar_roadway_02}
%  \caption[Solar Roadway in der Dämmerung]{Solar Roadway: In der Dämmerung wird der Weg an den nötigen Stellen beleuchtet}
%  \footnotesize Quelle: \url{http://static.boredpanda.com/blog/wp-content/uploads/2014/11/van-gogh-starry-night-glowing-bike-path-daan-roosengaarde-2.jpg}, Stand: 05.11.2015
%  \centering
%  \includegraphics[scale=0.5]{98_Bilder/03_Marktsegmente/solar_roadway_01}
%  \caption[Solar Roadway bei Nacht]{Solar Roadway: Bei Nacht ist der Veloweg komplett beleuchtet}
%  \footnotesize Quelle: \url{http://static.boredpanda.com/blog/wp-content/uploads/2014/11/van-gogh-starry-night-glowing-bike-path-daan-roosengaarde-1.jpg}, Stand: 05.11.2015
%\end{figure}
\newpage

\subsection{Detailhandel}
Im Verkauf hat die Revolution schon teilweise begonnen. Die grossen Unternehmen in der Schweiz beginnen alle ihre Filialen mit \gls{WLAN} auszustatten. Momentan bieten diese den \gls{WiFi} Zugang zur freien Verfügung den Kunden an. Somit steht der Kommunikationskanal für das \gls{IoT} im Laden bereit und die Kunden sind zur gegebenen Zeit bereits verbunden damit.
Weiter werden heutzutage Selbstbezahlkassen eingesetzt. Momentan werden zwei Modelle verfolgt. Eine Einkaufsart ist: Der Kunde wählt seine Produkte und scannt diese selber ein, bezahlt mit Karte oder Bar und verlässt das Geschäft. Bei der \gls{IoT} näheren Methode, registriert sich der Konsument beim Eingang an einem Terminal und rüstet sich mit einem mobilen Strichcodeleser der Filiale aus oder nimmt sein Smartphone als Scanner. Die Person liest alle Produkte mit dem Scanner ein und bezahlt, beim verlassen des Ladens, am Bezahlterminal. Beim abmelden des Scanners wird der Kunde ermittelt und das Total eingefordert.\\
Ein weiterer Schritt ist hier, alle Produkte mit einer \gls{RFID} zu taggen. Somit könnte der Kunde nur seine Kreditkarte registrieren, die Ware in den Einkaufskorb legen und die Filiale verlassen. Beim verlassen wird durch die Information auf dem \gls{RFID} Tag gemerkt, welche Ware mitgenommen wurde und die Kreditkarte wird automatisch belastet.
%\begin{figure}[H]
%  \centering
%  \includegraphics[scale=0.81]{98_Bilder/03_Marktsegmente/self_checkout}
%  \caption[Self Checkout Kasse]{Eine Selbstbezahlkasse für mit mobilen Scanner oder zum selber einlesen}
%  \footnotesize Quelle: \url{https://www.cooperation.ch/site/presse/get/12946772/Sutter-Selfscan_198B0923.jpg}, Stand: 05.11.2015
%\end{figure}

\section{Marktsegmente für Smartwatches}
\begin{tabbing}
xxxxxxxxxxxxxxxxxxxx\=xxxxxxxxxxxxxxxxxxxxxxxx	\kill
Mensch:		          \> Blutdruck, Puls, Bewegungen, Schlafüberwachung, Lebensüberwachung, \\\>Sportbeobachtungen, Sporttracking \\
Zeit:			          \> Individuelle Zeitansichten, Zeitfunktionen \\
Benachrichtigung:	  \> Informationen am Handgelenk, Kommunizieren
\end{tabbing}
Momentan werden Smartwatches hauptsächlich zu Notifikationszwecken und Fitnesstracking des Menschen genutzt. Noch wird das Potenzial nicht ausgenutzt Smartwatches in vielen anderen Segmenten einzusetzen. Um dies zu ermöglichen müssen die Bedürfnisse zuerst erkannt oder geschaffen werden.

\subsection{Mensch}
Heute werden Smartwatches verwendet, um den Menschen bei Aktivitäten überwachen zu können. Diese Wearables verfügen viele eingebaute Sensoren, die die Bewegungen des Trägers analysieren und dem interessierten die Daten zur Verfügung stellen. Zu den Sensoren gehören z.B. ein Bewegungssensor, Schrittzähler, Herzfrequenzmesser und viele mehr. Viele Hersteller von Smartwatches rüsten Ihre Produkte mit Sensoren und mit Auswertungsapplikationen (z.B. Google Fit, Apple Health oder Motorola Moto Body) aus. Mit diesen Apps kann der User seine Daten während dem Training auf der Uhr verfolgen oder später auf dem Smartphone auswerten. Dies macht zusätzliche Sport-/Pulsuhren überflüssig.

\subsection{Zeit}
Die Hauptaufgabe einer Uhr ist es, die Uhrzeit genau anzuzeigen. Die Smartwatches haben nicht nur die Möglichkeit die aktuelle Uhrzeit anzuzeigen, sondern auch als Weltuhr, Stoppuhr und Countdown-Rechner zu fungieren. Dabei hat der Träger der Computeruhr die Wahl, wie das Ziffernblatt aussehen soll. Wie individuell sie gestaltet werden wird vom Benutzer oder Entwickler bestimmt. Für Individualisten ist sie sehr geeignet.
%\begin{figure}[H]
%  \centering
%  \includegraphics[scale=0.25]{98_Bilder/03_Marktsegmente/smartwatchfaces}
%  \caption[Smartwatch Ziffernblätter]{Das Ziffernblatt der meisten Smartwatches kann individuell gestaltet werden}
%  \footnotesize Quelle: \url{http://i1-news.softpedia-static.com/images/news2/Google-Launches-Watch-Face-API-You-Can-Customize-Your-Smartwatch-467130-2.jpg}, Stand: 12.11.2015
%\end{figure}

\subsection{Benachrichtigung}
Eine Smartwatch wird neben der Uhrzeitfunktion auch als Notifikationsbildschirm verwendet. Alle relevanten Benachrichtigungen an ein Smartphone können auch von der Smartwatch angezeigt werden. Dabei dient die Uhr meist als verlängerter Arm des Mobilgerätes. Es können Nachrichten empfangen, Telefonate geführt, Erinnerungen ausgelöst, der Wecker gestellt werden oder anzeigen was auf dem Smartphone ausgeführt wird, z.B aktuell abgespieltes Musikstück.
%\begin{figure}[H]
%  \centering
%  \includegraphics[scale=0.3]{98_Bilder/03_Marktsegmente/notifications}
%  \caption[Smartwatch Anzeige von Smartphone]{Die Smartwatch zeigt an, welches Musikstück auf dem Smartphone abgespielt wird}
%  \footnotesize Quelle: \url{http://smartwatchpro.it/wp-content/uploads/2015/08/OB-YT760_smartd_M_20130904020012.jpg}, Stand: 12.11.2015
%\end{figure}
%\newpage

\section{Marktsegmente für Smartwatches im Internet of Things}
\begin{tabbing}
xxxxxxxxxxxxxxxxxxxx\=xxxxxxxxxxxxxxxxxxxxxxxx	\kill
Mensch:		        \> Blutdruck, Puls, Bewegungen, Schlafüberwachung, Gesundheitsbenachrichtigung \\
Benachrichtigung:	\> Alarme, Informationen \\
Heimautomation:	  \> Fernbedienung, Statusanzeigen, Alarming \\
Detailhandel:		  \> Geldbörse, Produktebezeichnung, Einkaufsliste \\
Ortsbezogen:		  \> Navigation, Ortsspezifische Informationen, Ortung, Personen in der Nähe \\
\end{tabbing}

\subsection{Mensch}
Um die Gesundheit eines Menschen zu überwachen, eignet sich eine Smartwatch sehr gut. Sie ist immer am Handgelenk und kann bei Unregelmässigkeiten Alarm schlagen. Durch die Vernetzung werden die Daten auch an anderen Geräten zur Verfügung gestellt. Eine wichtige Benachrichtigung kann von einer anderen Smartwatch oder einem Smartphone verwendet werden. Ein sich vorstellbares Szenario: Ein Paar, beide tragen eine smarte Uhr, bei einer potenziellen Gefahr, z.B. schwacher/kein Puls, Sturz, ungewöhnliche Bewegungsabläufe, wird der Andere alarmiert.

\subsection{Benachrichtigung}
Um Informationen darzustellen, eignet sich eine Smartwatch nur beschränkt so gut, wie ein Smartphone oder andere grössere Anzeigeapparate. Sie ist jedoch prädestiniert einzelne kleine Datenmengen anzuzeigen. Die im Internet der Dinge übermittelten und verwerteten Daten, können für den Menschen, auf einem praktisch sichtbaren Display am Handgelenk, angezeigt werden. Dies gewährt einen schnellen Zugriff zu den Benachrichtigungen.
%\begin{figure}[H]
%  \centering
%  \includegraphics[scale=1]{98_Bilder/03_Marktsegmente/infodisplay}
%  \caption[Smartwatch Anzeige von Tätigkeiten]{Die Smartwatch zeigt an, welche Herzfrequenz der Träger hat und wie lange er eine Tätigkeit ausführt}
%  \footnotesize Quelle: \url{https://support.apple.com/images/en_US/applewatch/watch-indoor-workout-heartrate.png}, Stand: 20.11.2015
%\end{figure}

\subsection{Heimautomation}
Im Smart Home Bereich kann die Kombination Smartwatch und Internet of Things ihre stärken ausspielen. Durch den Zusammenschluss aller Haushaltgeräte, wie Fernseher, Lampen oder Herdplatte, sind alle Daten zentral erreichbar und verwaltbar. Nun bestehen viele Möglichkeiten die Computeruhr ins System einzubinden. Einige geeignete Anwendungsfälle sind: Das Licht ein- und auszuschalten, Alarmierung von nicht ausgeschalteten Haushaltsgeräten oder das Fernbedienen von Geräten. All dies soll unabhängig vom Ort des Uhrträgers möglich sein.

\subsection{Detailhandel}
Im Detailhandel sind besonders Finanztechnologie Applikationen schon stark vertreten, wie Apple Pay, Google Wallet oder auch die schweizerische Lösung TWINT. Diese Anwendungen erlauben Geldüberweisungen mit Smartphones oder Smartwatches mittels drahtloser Verbindung über ein Terminal, welches die Bezahlung anfordert.\\
Des weiteren könnten Smartwatches in Selbstbedienungsgeschäften als erweiterter Informationsschild von Produkten dienen. Durch die Anbindung ans Internet hat man die Möglichkeit, jedes weitere Detail, welches nicht auf der Ware beschrieben ist, auf dem Bildschirm am Arm anzuschauen. Ein sehr grosser Vorteil ist, dass verschiedene Medien genutzt werden können, z.B. detailliert Nährwertangaben, Anleitungsfilme aller Art, Explosionszeichnungen von Modellen und viele weitere.

\subsection{Ortsbezogen}
Applikationen für die Ortung oder zur Anzeige standortabhängiger Daten werden bei vielen Anwendungen benutzt. Dies auf die Smartwatch zu erweitern ist ein logischer Schritt. Ein ortsrelevantes Thema kann direkt am Handgelenk angeschaut werden ohne das Smartphone heranzuziehen.
%\begin{figure}[H]
%  \centering
%  \includegraphics[scale=.5]{98_Bilder/03_Marktsegmente/watchmaps}
%  \caption[Smartwatch und ortsrelevante Informationen]{Landkarten anzeigen und ortsrelevante Informationen sind auf Smartwatches möglich}
%  \footnotesize Quelle: \url{http://icdn9.digitaltrends.com/image/here-maps-samsung-gear-s-720x480.jpg?ver=1}, Stand: 20.11.2015
%\end{figure}


% Main part - Part III
%---------------------------------------------------------------------------
\onecolumn
\chapter{Bedürfnisanalyse}
\section{Smartwatch Applikationen}
\subsection{Überwachung}
\textbf{Gesundheit}\\
Sturz erkennen:\\
Es wird ein Alarm ausgelöst, wenn nicht innerhalb von ca. 30s Bestätigung erfolgt und keine Bewegungen stattfinden

Puls überwachen:\\
Auch hier kann Alarmiert werden, wenn keine der Puls zu niedrig/hoch ist und vom Träger keine Aktionen erfolgen\\

\textbf{Alarming}\\
Spital: \\
Pflegeperson kann Patientenalarm direkt auf die Smartwatch erhalten

Haushalt: \\
Geräte im Haushalt können überwacht werden. Dies hilft Gefahren abzuwenden sowie Zeit zu optimieren.
z.B. Wenn eine Herdplatt noch läuft wird ein Alarm ausgelöst.
Oder wenn die Waschmaschine ihren Waschgang beendet hat kann der Träger dirket benachrichtigt werden.

\textbf{Sport}\\
Bewegungen: \\
Die getätigten Bewegungen beim Sport aufzeichnen mit den vorhandenen Sensoren.
Körper-Belastung messen wie z.B. Beschleunigung, Geschwindigkeit, Stärke usw.

\subsection{Fernbedienung}
Smart Home: \\
Das Fernbedienen von Geräte im Haushalt dürfte eine der interessantesten Anwendungsbereiche sein.
Da sind unbegrenzte Möglichkeiten vorhanden. Man kann das Licht steuern, nicht nur ein und ausschalten sondern auch gleich dimmen und timen.
Multimediageräte wie der TV oder die Set-Top-Box können aus dem Handgelenk gesteuert werden.
Mit der haptischen Rückmeldung, könnten sogar sehbehinderte Menschen davon profitieren.

\subsection{Umgebung}
Reminder:\\
Wen die Uhr aus dem Sichtbarkeitsumfeld des Smartphones gelangt, kann der Träger informiert werden.\\
Durch das erreichen einer Geofencing Zone können Erinnerungen ausgelöst werden.

Radar:\\
Es können Leute in der nähe ermittelt werden. Dies kann zu verschiedenen Zwecken eingesetzt werden.\\
Datingportale können diese Funktionen interessant einsetzen. Potenzielle Datingpartner können gleich mit Foto auf der Smartwatch angezeigt werden.
Die daraus resultierenden Kontakt Möglichkeiten wären, direkt anchatten, auf sich aufmerksam machen, ignorieren uvm.

\subsection{Navigation}
Indoornavigation:\\
In Zusammenarbeit mit Beacons/Eddystones und/oder Access Points können jeweils die Standorte von den Träger der Smartwatch ermittelt werden.
Dies ermöglicht Grossfirmen, die Mitarbeiter sich zu finden ohne direkten Kontakt zu haben.
Auch das Problem mit den Shared-Desk Arbeitsplätzen, dass diese meist besetzt sind man nicht weiss wo der nächste freie Platz ist, kann gelöst werden.
Mit der Smartwatch kann man sich bei einem freien Arbeitsplatz sich anmelden und diesen reservieren, nur durch erreichen des Schreibtisches.

\subsection{Authentifikation}
Türen:\\
Um aller Art Türen zu entriegeln können Smartwatches gebraucht werden.

Zugangskontrollen:\\
Die Smartwatch hat das Potenzial Personalausweise zu ersetzen. Zeitgleich kann es auch zu Zeiterfassung genutzt werden.
Der Mitarbeiter muss nicht mehr an die Zeiterfassungsleser, eintretten und austretten der Arbeitsumgebung kann automatisch erkannt und erfasst werden.

\subsection{Finanztechnologie - FinTech}
Zahlungen:\\
Die Möglichkeit nur mit der Uhr zu zahlen besteht. Es gibt bereits Lösungen welche mit Smartphones funktionieren {(Twint/Apple Pay/Google Wallet)}.
Diese Funktionen können auch auf die Smartwatch erweitert werden.

\section{Smartwatch Applikationen für siot.net}

Die siot.net Plattform bietet sich bestens als Kommunikationsschnittstelle an für die Applikationen, welche im vorherigen Abschnitt ermittelt wurden.
Somit sollte jede dieser Applikationen problemlos mit siot.net verknüpft werden können.

Um Verknüpfungen verschiedenster Applikationen mit einer Plattform zu erstellen sollte es eine generische Biblithek geben.
Diese sollte eine einfache Schnittstelle von der Applikation zu siot.net Plattform implementieren.

\subsection{siot.net Android Gateway Library}

TODO

\subsection{siot.net Dashboard App}

TODO



% Main part - Part IV
%---------------------------------------------------------------------------
\onecolumn
\chapter{Technische Anforderungen}
In diesem Kapitel werden die Technischen Voraussetzungen, für die, in der Bedürfnisanalyse ermittelten Anwendungen, definiert. Dabei wird unterschieden zwischen Allgemeine Applikationen, bei welcher nur eine kleine Auswahl beachtet wird, und zwischen siot.net Applikationen.

\section{Allgemeine Applikationen Anforderungen}
Die Technischen Anforderungen für die Bedürfnisse Gesundheit, Smart Home und Finanztechnologie werden in diesem Abschnitt definiert.

\subsection{Gesundheit}
Bei den Gesundheitsapplikationen ist es wichtig, dass die Smartwatch über einen Herzfrequenzmesser verfügt und die Kombination aus Gyroskop, Rotationssensor und Bewegungssensor eingebaut ist.
Diese Elemente sind erforderlich, um Pulsraten eines Menschen zu messen, sowie Analysen von Bewegungen zu erstellen.

\textbf{Benötigte Sensoren:}\\
- Gyroskop\\
- Bewegungssensor\\
- Rotationssensor\\
- Herzfrequenzmesser

\subsection{Smart Home}
Für die Steuerung von Haushaltsgeräten, Heizung oder Licht, wird vorallem eine Steuerungsanzeige benötigt. Diese ermöglicht die Apparate ein- und auszuschalten, die Beleuchtung zu dimmen und Benachrichtigungen (z.B. von offenem Fenster, beendetem Waschgang uvm.) anzuzeigen. Für eine automatische Helligkeitsregulierung sollte ein Lichtmesser eingebaut sein. Um Alarme ausgeben zu können, ist ein Lautsprecher oder Vibrationsmotor hilfreich.

\textbf{Bedienelement:}\\
- Touchscreen

\textbf{Aktoren:}\\
- Lautsprecher\\
- Vibrationsmotor

\textbf{Benötigter Sensor:}\\
- Lichtsensor

\subsection{Finanztechnologie - FinTech}
Um Geldtransaktionen durchführen zu können, wird eine drahtlose Verbindungseinheit vorausgesetzt. Bereits erwähnten Applikationen, wie Apple Pay und Google Wallet verwenden den \gls{NFC} (Near Field Communication) Chip und TWINT verwendet \gls{BLE}. Die Kommunikationsschnittstelle ist notwendig um Zahlungsanforderungen zu erhalten und diese auch zu autorisieren. Zum auswählen von Zahlungsoptionen und bestätigen von Überweisungen ist ein Bedienelement notwendig. Wenn mehrere Kreditkarten hinterlegt sind, muss der Anwender die Möglichkeit haben, eine bestimmte Kreditkarte für die Überweisung auszuwählen.

\textbf{Auswahl von benötigten Kommunikationsschnittstellen:}\\
- \gls{NFC} Chip\\
- Bluetooth\\
- \gls{GSM}/\gls{UMTS}/\gls{LTE}\\
- \gls{WLAN} (weniger geeignet)

\textbf{Mögliche Bedienelemente:}\\
- Touchscreen (beste Vorraussetzung)\\
- Mikrofon zur Sprachsteuerung\\
- Physische Taste

\section{siot.net Applikationen Anforderungen}
Technische Voraussetzung für die siot.net Applikationen werden für alle, in der Bedürfnisanalyse ermittelten Klassen definiert.

\subsection{siot.net Gateway Library}
Die Bibliothek, welche smarte Geräte an die siot.net-Plattform anbinden soll, wird in erster Linie nur für das Android Betriebssystem entwickelt. Für die Benutzung von Apps, welche ans siot.net angeschlossen werden, müssen diese mindestens die Android Tools der \gls{SDK} Version 21 beherrschen (ab Android 5.0 Lollipop). Um die Daten an den \gls{MQTT} Broker übermitteln zu können, braucht es ein Netzwerkmodul, dass eine Verbindung ins Internet erlaubt. Die Bibliothek sollte alle verfügbaren und bekannten Sensoren selber erkennen. Die Verwendung, ob Sensoren aktiviert werden oder nicht, wird durch den Appentwickler oder der Software selber definiert.

\textbf{Auswahl von benötigten Kommunikationsschnittstellen:}\\
- \gls{NFC} Chip\\
- Bluetooth\\
- \gls{GSM}/\gls{UMTS}/\gls{LTE}\\
- \gls{WLAN}

\textbf{Betriebssystem:}\\
- ab Android 5.0\\
- ab \gls{SDK} Tools Version 21 (Android Tools)

\subsection{siot.net Sensorcenter}
Mit der Sensorcenter Applikation von siot.net, soll jeder beliebige Benutzer die Sensormesswerte seines Android Gerätes ans siot.net senden können. Um dies zu ermöglichen muss eine Anmeldung ans siot.net, manifestieren von Sensoren, sowie senden von Messungen möglich sein. Diese App kann vom Bestehen der siot.net Gateway Library profitieren, welche integriert werden soll. Technisch benötigt diese Applikation, zusätzlich zu den Voraussetzungen der siot.net Gateway Library, ein Touchscreen.

\textbf{Auswahl von benötigten Kommunikationsschnittstellen:}\\
- \gls{NFC} Chip\\
- Bluetooth\\
- \gls{GSM}/\gls{UMTS}/\gls{LTE}\\
- \gls{WLAN}

\textbf{Bevorzugtes Bedienelement:}\\
- Touchscreen

\textbf{Betriebssystem:}\\
- ab Android 5.0\\
- ab \gls{SDK} Tools Version 21 (Android Tools)

\subsection{siot.net Dashboard App}
Um Sensordaten von der siot.net-Plattform darzustellen, eignet sich eine Dashboard App. Anforderungen für diese Applikation sind in dieser Arbeit nur für Android Geräte spezifizert. Für die Darstellung einer derartigen digitalen Instrumententafel eignet sich ein Display, bevorzugt ein Touchscreen. Ein Berührbildschirm erlaubt es die gewünschten Anzeigen bequem einzublenden. Eine weitere Voraussetzung ist die Vernetzung. Das Gerät muss einen Zugang zum Internet herstellen können. Nur durch eine erfolgreiche Anmeldung an den siot.net \gls{MQTT} Broker, können die Informationen empfangen werden.

\textbf{Auswahl von benötigten Kommunikationsschnittstellen:}\\
- \gls{NFC} Chip\\
- Bluetooth\\
- \gls{GSM}/\gls{UMTS}/\gls{LTE}\\
- \gls{WLAN}

\textbf{Bevorzugtes Bedienelement:}\\
- Touchscreen

\textbf{Betriebssystem:}\\
- ab Android 5.0\\
- ab \gls{SDK} Tools Version 21 (Android Tools)

\subsection{Herzfrequenzüberwachung}
Die Herzfrequenzüberwachung ist eine schlanke Variante des Dashboards. Um eine Anzeige, wie bei einem Elektrokardiogramm (\gls{EKG}) darzustellen, braucht es ein Display. Und um Messdaten zu erhalten braucht es einen Pulsmesser. Für einen Alarm auszulösen, beim Überschreitung von definierten Werten, braucht es einen Lautsprecher und/oder einen Vibrationsmotor, damit akustische oder taktile Signale ausgesendet werden können.

\textbf{Auswahl von benötigten Kommunikationsschnittstellen:}\\
- Bluetooth\\
- \gls{GSM}/\gls{UMTS}/\gls{LTE}\\
- \gls{WLAN}

\textbf{Bevorzugtes Anzeigen / Aktoren:}\\
- Touchscreen\\
- Lautsprecher\\
- Vibrationsmotor

\textbf{Sensoren:}\\
- Herzfrequenzmesser

\textbf{Betriebssystem:}\\
- ab Android 5.0\\
- ab \gls{SDK} Tools Version 21 (Android Tools)

\subsection{Steuerung von Modellen}
Um die Kontrolle über ein Modellfahrzeug oder Modellflugzeug zu erhalten, benötigt es einen Hardwarekontroller, welcher sich mit dem siot.net \gls{MQTT} Broker verbinden kann. Die Steuerungseinheit muss auf die \gls{MQTT} Topic subscriben und die empfangenen Bewegungsdaten interpretieren und an die erforderlichen Antriebsmotoren und Servos weitergeben werden. Als Kontroller könnte ein Raspberry Pi Zero oder Ähnliches zum Tragen kommen. Auf den Hardwarekontroller wird nicht weiter eingegangen. Als Sender und Ermittler der Bewegungs- und Beschleunigungsdaten kommt ein Android Device (z.B. mit siot.net Sensorcenter) zum Einsatz. Damit muss die Spezifikation für die Steuerung nicht mehr weiter betrachtet werden, da diese mit dem Sensorcenter abgedeckt wird.

\textbf{Auswahl von benötigten Kommunikationsschnittstellen:}\\
- Bluetooth\\
- \gls{GSM}/\gls{UMTS}/\gls{LTE}\\
- \gls{WLAN}

\textbf{Bevorzugtes Anzeigen / Aktoren:}\\
- Touchscreen\\
- Lautsprecher

\textbf{Sensoren:}\\
- Bewegungssensor\\
- Beschleunigungssensor\\
- Gyroskop

\textbf{Betriebssystem:}\\
- ab Android 5.0\\
- ab \gls{SDK} Tools Version 21 (Android Tools)

\newpage



% Main part - Part 5V
%---------------------------------------------------------------------------
\onecolumn
\chapter{Technologiewahl}
Im Kapitel der Technologiewahl werden einige aktuell erhältliche Smartwatch Modelle mit den theoretischen Daten verglichen, allegemeine Eigenschaften aufgeführt und die Entwicklerfreundlichkeit bewertet. Aufgrund der Bewertung von aufgestellten Kriterien werden ein bis zwei Uhren ausgewählt für die Arbeit.

\section{Allgemeine Eigenschaften}
Heute (Januar 2016) sind weitgehends alle erhältlichen tragbaren Computer am Handgelenk vom Smartphone abhängig, deshalb bieten sie nur einen kleinen Mehrwert. Sie können zwar viel mehr als nur die Zeit anzeigen, jedoch sehr beschränkt und zum grössten Teil wird das Mobiltelefon benötigt. Nachrichten, Termine und einkommende Telefonate anzeigen, aber ein Gespräch kann nur mit den wenigsten Geräten geführt werden, wie z.B. der Apple Watch. Mit den meisten Smartwatches können Nachrichten gelesen, jedoch nicht beantwortet werden, da diese über keine virtuelle Tastatur verfügen. Falls doch geschieht dies hauptsächlich über die Spracheingabe. Mit dem integrierten Touchscreenkeyboard, stellt die Samsung Gear S dar die einzige Ausnahme dar.

Smartwatches wollen die Gesundheit fördern, etliche davon messen den Puls, zählen die Schritte und erfassen zurückgelegte Wegstrecke. Weitere Apps erlauben den Schlaf zu überwachen, zur Bewegung aufzufordern bei langer inaktivität und den Kalorienbedarf und Verbrauch berechnen. Durch die genauen Daten haben Amateursportler eine praktische Hilfe am Handgelenk, um ihren Körper fit zu halten. Für ambitionierte Sportler ist der Funktionsumfang no zu gering.

Manche Uhren der Hersteller Apple, LG, Samsung und Alcatel messen den Puls fast Elektrokardiogramm-genau. Apple arbeitet bereits an einem Elektrokardiogramm (EKG) Armband, welches Daten per Ultraschall übermitteln soll. Das Band ermittelt die Herzströme über zwei angebrachte Elektroden. Dies ermöglicht der Apple Smartwatch als mobiles EKG bereitzustehen\footnote{vgl. \url{http://www.smartwatch.de/news/neues-apple-watch-armband-mit-ultraschall-ekg-geplant}, 20.11.2015}.

Ein grosse, fast alle betreffende, Schwachstelle von Smartwatches sind ihre leistungsschwachen Akkumulatoren. Durchschnittlich erreichen die Uhren eine Laufzeit von knapp 24 Stunden bei geringer Nutzung.

Applikationen welche einen grossen Mehrwert bringe gibt es no zu wenige. Zum heutigen Zeitpunkt stellt sich heraus, dass die Minicomputeruhren überwiegend als Erweiterungsbildschrim des Smartphones dient. Wenige Geräte werkeln autonom, wie zum Beispiel die Samsung, jedoch bleiben die meisten ein verlängerter Arm des dazugehörigen Mobiltelefons. Dies liegt daran, weil die smarten Uhren noch Heute in den Kinderschuhen stecken.

\section{Kandidaten}
\begin{table}[H]
\begin{minipage}{\textwidth}
\centering
\begin{tabular}{|>{\columncolor[gray]{0.8}}p{4cm}|p{4cm}|p{4cm}|p{4cm}|}
\hline

  & \includegraphics[width=2.5cm]{98_Bilder/06_Smartwatch_Produkte/AppleWatch}
  & \includegraphics[width=2.5cm]{98_Bilder/06_Smartwatch_Produkte/LGGWatch}
  & \includegraphics[width=2.5cm]{98_Bilder/06_Smartwatch_Produkte/LGWatchUrbane} \\ \hline
\textbf{Smartwatch}
  & Apple Watch\footnote{vgl. Stiftung Warentest Magazin, Ausgabe: 10/2015}
  & LG G Watch\footnote{vgl. Modul BTI7302 Evaluation Wearables, 27.06.2015}
  & LG Watch Urbane\footnote{vgl. Stiftung Warentest Magazin, Ausgabe: 10/2015} \\ \hline
\textbf{Betriebsystem}
  & WatchOS
  & Android Wear
  & Android Wear \\ \hline
\textbf{Funktionen}
  & gut
  & befriedigend
  & befriedigend \\ \hline
Nachrichten
  & empfangen möglich, \newline senden per Sprachnachricht, \newline Emojis senden
  & empfangen möglich, \newline senden per Sprachnachricht
  & empfangen möglich, \newline senden per Sprachnachricht \\ \hline
Telefonieren
  & führen von Gespräche
  & nur Notifikation
  & nur Notifikation \\ \hline
Display
  & 42mm 312x390px \newline 38mm 272x340px
  & 1.65 Zoll 280x280px
  & 1.3 Zoll 320x320px \\ \hline
\textbf{$\varnothing$ Akkulaufzeit}
  & 19h
  & 30h
  & 20h \\ \hline
\textbf{Datensendungsverhalten}
  & verschlüsselt
  & verschlüsselt
  & verschlüsselt \\ \hline
\textbf{Smartphone-Betriebssystem}
  & ab iOS 8.2
  & ab Android 4.3 \newline ab iOS 8.2
  & ab Android 4.3 \newline ab iOS 8.2 \\ \hline
\textbf{Pulsmesser}
  & optischer Pulsmesser
  & kein Pulsmesser
  & optischer Pulsmesser \\ \hline
\textbf{Gesamteindruck}
& bestes Display \newline viele Funktionen \newline \textbf{befriedigend - gut}
& Always-On Display \newline befriedigender Akku \newline \textbf{befriedigend}
& rundes Display \newline edle Verarbeitung \newline \textbf{befriedigend} \\ \hline
\end{tabular}
\caption{Technische Daten von Smartwatches - Teil 1}
\end{minipage}
\end{table}

\begin{table}[H]
\begin{minipage}{\textwidth}
\centering
\begin{tabular}{|>{\columncolor[gray]{0.8}}p{4cm}|p{4cm}|p{4cm}|p{4cm}|}
\hline

  & \includegraphics[width=2.5cm]{98_Bilder/06_Smartwatch_Produkte/MotorolaMoto360}
  & \includegraphics[width=2.5cm]{98_Bilder/06_Smartwatch_Produkte/SamsungGearS2}
  & \includegraphics[width=2.5cm]{98_Bilder/06_Smartwatch_Produkte/AlcatelOnetouchWatch} \\ \hline
\textbf{Smartwatch}
  & Motorola Moto 360\footnote{Referenz: Modul BTI7302 Evaluation Wearables, Michael Fankhauser und Sathesh Paramasamy, Ausgabe: 27.06.2015}
  & Samsung Gear S2\footnote{Referenz: \url{https://www.androidpit.de/samsung-gear-s2-test}, Stand: 27.12.2015}
  & Alcatel Onetouch Watch\footnote{Referenz: Stiftung Warentest Magazin, Ausgabe: 10/2015} \\ \hline
\textbf{Betriebsystem}
  & Android Wear
  & Tizen
  & proprietäres Alcatel OS \\ \hline
\textbf{Funktionen}
  & befriedigend
  & gut
  & befriedigend \\ \hline
Nachrichten
  & empfangen möglich, \newline senden per Sprachnachricht
  & empfangen möglich, \newline senden möglich
  & empfängt nur Kurznachrichten \\ \hline
Telefonieren
  & nur Notifikation
  & nur Notifikation \newline bei eSIM Variante möglich
  & nur Notifikation \\ \hline
Display
  & 1.57 Zoll 360x290px
  & 1.2 Zoll 360x360px
  & 1.3 Zoll 320x320px \\ \hline
\textbf{$\varnothing$ Akkulaufzeit}
  & 10h
  & 38h
  & 20h \\ \hline
\textbf{Datensendungsverhalten}
  & verschlüsselt
  & verschlüsselt
  & unverschlüsselt \\ \hline
\textbf{Smartphone-Betriebssystem}
  & ab iOS 8.2
  & ab Android 4.4
  & ab Android 4.3 \newline ab iOS 7 \\ \hline
\textbf{Pulsmesser}
  & optischer Pulsmesser
  & optischer Pulsmesser
  & optischer Pulsmesser \\ \hline
\textbf{Gesamteindruck}
& bestes Display \newline meiste Funktionen \newline \textbf{befriedigend}
& Intuitive Lünette \newline gute Akkuleistung \newline \textbf{befriedigend - gut}
& schwaches Display \newline mangelhafte Datenübertragung \newline \textbf{mangelhaft} \\ \hline
\end{tabular}
\caption{Technische Daten von Smartwatches - Teil 2}
\end{minipage}
\end{table}

\section{Entscheid Smartwatch}
Wenn der Funktionsumfang betrachtet wird, wäre eine Apple Watch oder die Samsung Gear S2 die bevorzugte Wahl gewesen. Die Samsung ist erst seit Oktober 2015 erhältlich, folglich fiel diese Aufgrund des späten Erscheinungsdatum und dessen Samsung proprietären Betriebssystem Tizen aus dem Katalog. Die Apple Watch wurde nicht gewählt, weil die Entwicklung sich auf die Apple spezifische Programmiersprache Swing beschränkt. Ein weitere Entscheidungspunkt ist, dass der Markt aus einem Produkt besteht, wenn man die Modelle Watch und Watch Sport und die Grössen, 38mm und 42mm, nicht unterscheidet, da diese sich nur optisch voneinander unterscheiden.
Ausgewählt wurden zwei Produkte, welche nicht sehr aufgefallen sind: Die \textbf{LG G Watch} und die \textbf{Motorola Moto 360}. Als erste auf dem Markt erhältliche Android Wear Smartwatch, wurde die LG G Watch ausgewählt. Der eingebaute Snapdragon 400 Prozessor rechnet mit 1,2 GHz, hat 512MB RAM und hat ein rechteckigen Bildschrim, welcher die Funktion biete immer eingeschaltet zu sein. Wichtiger Pluspunkt ist, die LG Datenuhr kann über USB gedebugged werden. Dies erleichtert die Entwicklung von Applikation. Um ein Android Wear Vergleichsobjekt zu erhalten, kam die neuere Motorola Moto 360 zusätzlich in die Entscheidung. Diese arbeitet mit einem Texas Instrument OMAP 3 Prozessor mit 1 GHz, hat ebenfalls 512MB RAM und hat einen runden Touchscreen. Mehrwert der Moto 360 zur G Watch sind, ein eingebautes WLAN Modul, Lichtsensor, Pulssensor und eine physische Taste. Bedauerlicherweise kann sie nur über Bluetooth gedebbugt werden, was viel Zeit beansprucht um entwicklte Apps zu übertragen. Durch die Entscheidung für Android Wear liegt der Schwerpunkt auf der Entwicklungsprache JAVA.


%---------------------------------------------------------------------------

%---------------------------------------------------------------------------

% Index
%---------------------------------------------------------------------------
%\cleardoublepage
%\phantomsection
%\addcontentsline{toc}{chapter}{Stichwortverzeichnis}
%\renewcommand{\indexname}{Stichwortverzeichnis}
%\printindex
%---------------------------------------------------------------------------

%---------------------------------------------------------------------------
\end{document}
